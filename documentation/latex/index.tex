{\bfseries I\+UT Réseau et Télécom}

{\itshape projet par Benjamin et Maeg} 



{\bfseries Le fichier Make s\textquotesingle{}occupe de générer de la documentation via doxygen}

\subsection*{Partie 1}

\subsubsection*{Conception du jeu}

Le jeu à pour objectif la cueillette de champignon\+: Le joueur est un cueilleur qui se déplace sur un plateau afin de récolter des champignons tout en évitant les dangers.

Les différentes entités sur le plateau de jeu sont\+:
\begin{DoxyItemize}
\item Le cueilleur Personnage que l\textquotesingle{}on déplace sur des axes x et y dans un tableau pouvant bénéficier d\textquotesingle{}effets de la part des champignons qu\textquotesingle{}il ramasse.
\item Le champignon Entité que le cueilleur ramasse ou que les sangliers piétinent, il reste fixe. Peut affecter différents effets au cueilleur.
\item Le sanglier Sa vitesse peut varier de 1 à 2 cases par tour. Lorsqu\textquotesingle{}il passe sur un champignon ce dernier est alors piétiné. 
\end{DoxyItemize}